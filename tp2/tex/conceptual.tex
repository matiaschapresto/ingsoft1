\subsection{Modelo Conceptual}
\subsubsection{Diagrama de clases}
\diagramav{Modelo-Conceptual}
\subsubsection{Aclaraciones}
\begin{itemize}
	\item La herencia es siempre disjunta y completa.
	\item La clase \texttt{Persona} fue agregada con el objetivo de unificar los atributos de diferentes subclases de \texttt{Persona}.
	\item \textbf{Notar que las herencias pueden ser varios conectivos de herencia. ie. Ver clase Persona, de la cual heredan Empleado y Pasajero.}
\end{itemize}
\subsubsection{Restricciones al modelo con OCL}
\begin{enumerate}
	\item \begin{ocl}{Puntaje del taxista entre 0 y 10.}
		  context Taxista
		  inv: self.puntajePromedio >=0 and
		       self.puntajePromedio <=10
		\end{ocl}
	
	\item \begin{ocl}{Puntaje del viaje entre 0 y 10. o null}
		  context Viaje
		  inv: self.puntaje >=0 and
		       self.puntaje <=10
		\end{ocl}

	\item \begin{ocl}{No puede haber 2 viajes con el mismo taxista en el mismo momento.}
		  context Viaje
		  inv: self.allInstances() -> not(exists(v1, v2 | 
		     (v2.fechaFinal >= v1.fechaComienzo and v2.fechaFinal <= v1.fechaFinal) or 
		     (v2.fechaComienzo >= v1.fechaComienzo and v2.fechaComienzo <= v1.fechaFinal)
		     and v1.Taxista == v2.Taxista))
		\end{ocl}
		\textbf{En palabras: No pueden existir dos viajes v1 y v2 con el mismo taxista y con la fecha de comienzo o de final de alguno de ellos incluida en el rango de lo q duró el viaje del otro.}

	\item \begin{ocl}{Si una Reserva tiene un viaje, el pasajero creador de la reserva debe ser el pasajero del viaje asociado a la reserva.}
		  context Reserva
		  inv: self.ViajeAsociado.Pasajero == self.creador
		\end{ocl}

	\item \begin{ocl}{Si una Reserva es en el futuro, el viaje asociado debe estar en estado pendiente.}
		  context Reserva
		  inv: self.FechaReserva >= today() implies self.ViajeAsociado.EstadoViaje == Pendiente
		\end{ocl}

	\item \begin{ocl}{El domicilio de origen y destino de un viaje debe ser distinto.}
		  context Viaje
		  inv: not(self.DomicilioOrigen == self.DomicilioDestino)
		\end{ocl}

	\item \begin{ocl}{La fecha comienzo de un viaje debe ser menor que la fecha fin.}
		  context Viaje
		  inv: self.EstadoViaje == Finalizado implies 
		  self.FechaComienzo < self.FechaFinal
		\end{ocl}

	\item \begin{ocl}{No hay usuarios repetidos por clave \texttt{nombre de usuario}.}
		  context Persona
		  inv: self.AllInstances() -> not(exist(p1, p2 | p1.NombreUsuario == p2.NombreUsuario))
		\end{ocl}	

	\item \begin{ocl}{La fecha de la planilla de facturaci\'on diaria a la que pertenece cada viaje coincide con la fecha del final del viaje.}
		  context Viaje
		  inv: self.FacturacionDelDia.Dia == self.FechaFinal
		\end{ocl}

	\item \begin{ocl}{El monto facturado de una clase Facturacion corresponde a la suma de todos los viajes que tiene la clase.}
		  context Viaje
		  inv: self.FacturacionDelDia.MontoFacturado == self.AllInstances()
		  ->select(p|p.fechaFinal == self.fechaFinal AND p.Costo <> 0)->collect(Costo)->sum()
		\end{ocl}
		

	\item \begin{ocl}{La fecha del historial de viajes que tiene un viaje, debe coincidir con la fecha de inicio del viaje.}
		  context Viaje
		  inv: self.Historial.Dia == self.FechaComienzo
		\end{ocl}

			\item \begin{ocl}{La suma del historial de viajes que tiene un viaje, debe coincidir con la cantidad de viajes de esa fecha.}
		  context Viaje
		  inv: self.Historial.cantViajes == self.AllInstances()
		  ->select(p|p.fechaComienzo == self.fechaComienzo)->count()
		\end{ocl}


	\item \begin{ocl}{La calificacion asociada a un taxista, debe contener el puntaje promedio que corresponde al promedio de puntaje de los viajes ya calificados de dicho taxista}
		  context Taxista
		  inv: let viajesPuntuados : Set(Viaje) = self.Viajes->select(v|v.puntaje <> 0) in
		  self.Calificacion.PuntajePromedio == 
		  viajesPuntuados->collect(Puntaje)->sum().div(viajesPuntuados->count())
		\end{ocl}
		

\end{enumerate}
