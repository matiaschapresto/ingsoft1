\section{Introducción}
El objetivo de este documento es describir los aspectos del sistema que hayan quedado afuera en especificaciones anteriores. Para ello, utilizaremos diversos diagramas y herramientas a nuestro alcance, tales como:
\begin{itemize}
	\item Una descripcion del sistema en lenguaje natural.
	\item Un diagrama de contexto donde se denotan los fenomenos entre los diversos agentes.
	\item Una lista de requerimientos, asociada con el diagrama de objetivos del sistema.
	\item Un diagrama de casos de uso, explicando la interfaz entre el usuario y el sistema. Ademas se incluyen las fichas con los detalles de todos los casos de uso especificados.
	\item Un diagrama de clases indicando ciertos aspectos del sistema, acompañado de una lista de invariantes formales en lenguaje OCL que nos permiten restringir las relaciones especificadas entre las distintas clases.
	\item Un conjunto de diagramas de actividad que modelan el comportamiento de distintas funcionalidades del sistema.
	\item Un conjunto de maquinas de estado finito que modelan otros aspectos del sistema que no quedaron cubiertos por modelos anteriores.
\end{itemize}