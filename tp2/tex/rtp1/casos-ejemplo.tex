\subsection{Descripcion General del Sistema: Casos de ejemplo}
	\subsubsection{Caso: Viaje exitoso}
	El pasajero desea viajar del punto A al punto B en auto. Para ello quiere contratar los servicios de TecnoTaxi, el primer paso que realiza es ingresar a su perfil via internet(o comunicarse con la operadora en caso de no estar familiarizado con la tecnologia) y solicitar un movil al sistema(o por medio de la operadora de ser necesario). El sistema computa una lista de taxistas disponibles y con caracteristicas deseadas por el pasajero, y se las presenta al pasajero, al este elegir un taxista, el sistema le comunica al taxista el nuevo viaje, en caso de que el taxista acepte, el pasajero es notificado que el movil esta en camino. El usuario puede verificar en el sistema(por medio de operadora o el sistema) el tiempo estimado de espera o cancelar el viaje. Cuando el taxista arriba y se realiza el viaje desde A hasta B, el taxista pide la cantidad adecuada de dinero y el pasajero entrega dicha cantidad(ver diagrama de objetivos para esta operacion). Al finalizar, el pasajero otorga una puntuacion al taxista y finaliza el servicio contratado por TecnoTaxi para este viaje.

	\subsubsection{Caso: Sin internet}
	Un usuario desea pedir un taxi para realizar un viaje, pero no posee ningún dispositivo con acceso a internet. Este se comunica telefónicamente con la operadora de RadioTaxi indicandole origen, destino y horario en el que desea el taxi. La operadora posee acceso directo a través de una red local al sistema, a través de una interfaz reducida carga manualmente los datos para el viaje y el sistema se encarga de asignarle el viaje a un taxista notificando al mismo. Una vez asignado el viaje, se comunica a la operadora el taxista encargado y esta a su vez comunica al cliente. En caso de no poseer ningún taxi disponible el sistema lo indicará y la operadora informará al cliente de la misma manera. Una vez que el taxista retire al pasajero, el resto del viaje continua con su curso normal.

	\subsubsection{Caso: Caida de conexión a internet y solución de contingencia}
	Durante un viaje el taxista deja de tener acceso a internet. La interfaz que este posee dentro de su veh\'iculo lo notifica y la comunicación durante este per\'iodo será exclusivamente a través de la radio con la operadora, indicando cuando concrete sus viajes y siendo notificado cuando le es asignado un viaje. La operadora se encargará de mantener la consistencia en el sistema cargando manualmente lo que cargar\'ia durante el transcurso de un viaje normal el taxista, es decir la finalización de los viajes, las transacciones de dinero, la cancelación de un viaje por no disponibilidad, etc.