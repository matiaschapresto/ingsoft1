\subsection{Descripcion General del Sistema}
TecnoTaxi es un sistema para facilitar la coordinación de los viajes entre los pasajeros y los taxistas.
\subsection{Descripcion General del Sistema: Pasajeros}
	\subsubsection{Interfaz con el usuario}
	El usuario podrá utilizar el sistema a través de dos interfaces distinguidas. La primera es de forma directa con TecnoTaxi, utilizando un dispositivo con acceso a internet. La segunda será a través de una operadora que servirá de nexo entre el usuario y el sistema.

	\subsubsection{Utilización del sistema a través de internet}
	Antes de tener acceso al sistema, es necesario registrarse. El usuario deberá completar nombre, apellido, y deberá señalar al menos un teléfono o e-mail para poder contactarlo e identificarlo unívocamente. El usuario deberá además indicar una contraseña para restringir el acceso a su cuenta. Una vez registrado tendrá acceso al sistema, donde podrá solicitar viajes ingresando una dirección de partida, dirección de destino y horario. El sistema responderá con un listado de taxis disponibles para la realización del viaje, luego el usuario procederá a seleccionar uno de estos y será notificado cuando le sea asignado un taxista para concretar su viaje (que podría no llegar a ser el seleccionado, por una posible cancelacion por parte del taxista). El usuario podrá cancelar el viaje a través del sistema en cualquier momento. Una vez finalizado el viaje, el usuario podrá puntuar al taxista con el que lo realizó. El listado de taxista tendrá además información específica del perfil del taxista, como su puntuación y el vehículo que utiliza.

	\subsubsection{Utilización del sistema a través de la operadora}
	El usuario podrá comunicarse telefónicamente con una operadora con acceso al sistema vía red local. Esto sirve como plan de contingencia en caso de que la conexión a internet este caída, y también sirve de utilidad para aquellos usuarios que no poseen conectividad a internet, o simplemente desean pedir un taxi por este medio. Todas las funcionalidades disponibles al usuario a través del sistema podrán realizarse también telefónicamente, la operadora humana lo guiará en todo el proceso, creandole un usuario si este no dispone de uno, o pidiéndole primero sus datos personales para tener acceso a la cuenta del mismo. La operadora podrá acceder sin restricción de contraseña a cualquier cuenta de usuario.

	\subsubsection{Interfaz con el usuario a través de internet}
	La web tendrá una página principal, donde el usuario podrá acceder o registrase al sistema ingresando sus datos. Una vez en el sistema, tendrá un menú de viajes donde podrá realizar las siguientes operaciones:
	\begin{enumerate}
	\item Agregar un nuevo viaje indicando punto de partida, destino y horario.
	\item Cancelar un viaje pendiente.
	\item Calificar un viaje finalizado si no fue calificado previamente.
	\item Crear reserva periódica de viajes.
	\item Cancelar una reserva pendiente.
	\end{enumerate}
	En el primer caso, en la misma pantalla se desplegará un menú con los taxistas recomendados por el sistema al usuario. Los taxistas que se mostrarán serán filtrados por puntuación general, puntuación personalizada, distancia, disponibilidad para la realización del viaje. Una vez seleccionado el taxista, TecnoTaxi tratará de asignarle a este el viaje, en caso fallido asignará otro de los taxistas por orden de prioridad según los criterios que se consideren relevantes (como la puntuación y la distancia). Una vez asignado el taxista, el viaje pasará a estar en estado pendiente.
	El usuario podrá cancelar cualquier viaje que se encuentre en estado pendiente. También una vez que se haya finalizado el viaje, podrá puntuar al taxista con el que lo realizó.
	Como indica el punto 4, el usuario también podrá programar viajes periódicamente estableciendo un patrón y la frecuencia con la que se quiere que este se repita. De la misma manera, podrá seleccionar un taxista preferido para la reserva, pero eso no es garantía de que el seleccionado sea el encargado de realizar el viaje. El usuario podrá abonar este servicio a través de un sistema de débito automático, o de la manera tradicional al finalizar cada uno de los viajes. Por ultimo, de la misma forma en la que puede cancelar un viaje, puede cancelar una reserva

\subsection{Descripcion General del Sistema: Taxistas}
	\subsubsection{Utilización del sistema por parte de los taxistas}
	\label{section:utilizaciontaxista}
	Todos los choferes de la empresa de RadioTaxi tendrán una cuenta con un perfil asociado en el sistema TecnoTaxi. Esta cuenta será de utilidad para asociar a este información como nombre, apellido, vehículo, puntuación y otros datos de interes para los potenciales pasajeros. Los taxistas tendrán una computadora con pantalla táctil integrada a sus vehículos, la misma estará asociada a su cuenta en TecnoTaxi. Esta se comunicará con el sistema utilizando tecnología de internet 3G. Además cada uno de los taxis tendrá un sistema de posicionamiento global (GPS), embebido que será de utilidad para saber la ubicación de los mismos. A través de esta los taxistas serán notificados cuando les es asignado un viaje, indicándoles la información relevante al viaje (horario, partida, destino). Poseerán un menú con el listado de viajes donde podrán realizar las siguientes acciones:
	\begin{enumerate}
	\item Marcar un viaje pendiente como en progreso.
	\item Marcar un viaje en progreso como finalizado.
	\item Marcar un viaje no finalizado como cancelado.
	\item Marcarse como no disponible.
	\end{enumerate}
	La primera de estas acciones se realizará cuando el taxista retire a un pasajero por un lugar, iniciando así el viaje. La segunda se dará cuando el viaje termine, una vez que clickee en finalizar se le indicará el costo del viaje para poder cobrar así al pasajero el monto correspondiente. El costo del mismo será calculado de la manera tradicional, en base a la distancia y el tiempo de duración del viaje. Un viaje no finalizado (asignado al taxista) podrá cancelarse en cualquier momento, por ejemplo si no pudo pasar a retirar al pasajero o si no podrá hacerlo. El último item consiste en una opción que tiene el taxista para marcarse como no disponible, esto puede ser de utilidad si el vehículo posee un desperfecto ténico y es necesario repararlo antes de poder continuar trabajando.

	\subsubsection{Utilización del sistema por parte de los taxistas a través de radio}
	En caso de que la computadora integrada en el taxi no pueda comunicarse con TecnoTaxi, el taxista será notificado y deberá comunicarse con la operadora a través de la radio para notificar de este suceso. De ese momento en adelante, el taxista deberá informar a la operadora cada vez que realiza una de las acciones determinadas en el punto anterior, y a su vez la operadora deberá informar al taxista de los viajes que debe realizar. La operadora se encargará de introducir manualmente estos datos en TecnoTaxi, utilizando la cuenta del taxista que se haya sin conexión, hasta que el servicio se normalize.

	\subsection{Descripcion General del Sistema: Otros aspectos del sistema}
	En caso de que se cancele un viaje, esto figurará en el menú de viajes del pasajero. Si el sistema puede, asignará automáticamente otro taxista.

	El taxista será libre de tomar el curso que quiera a la hora de realizar un viaje, por ejemplo, para evitar tráfico durante el mismo. De todas maneras se proveerá de un costo estimado al usuario una vez que seleccione el viaje, utilizando la información asociada a este.

	El sistema TecnoTaxi guardará toda la información relevante a puntuaciones, cancelaciones, viajes finalizados, costo de viajes, porcentaje de disponibilidad de taxistas en un momento dado, etc. Todas estos datos serán presentados en forma de estadísticas a través de un sistema interno a la empresa, alojado en una intranet, restringida por contraseña, para los gerentes de la empresa de radio taxi.