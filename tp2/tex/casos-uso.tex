\subsection{Casos de Uso}

\subsubsection{Diagrama de Casos de Uso}
\diagramav{diag-cu}

\subsubsection{Fichas de los CU}

\begin{usecase}
  \ucname{Registrandose en el Sistema}
  \ucpre{True}
  \ucpost{El pasajero queda registrado con una cuenta en el sistema}
  \ucactor{Pasajero}
\end{usecase}
\begin{usecasesteps}
  \ucstep{El pasajero ingresa sus datos(Ver modelo conceptual).}{}
  \ucstep{El sistema verifica que los datos ingresados sean validos.}{}
  \ucstep{Los datos son validos. El Sistema muestra la pantalla de logueo.}
        {\ucalt{Los datos no son validos, el sistema muestra el mensaje de error correspondiente. Ir a 1.}}
  \ucstep{Puede ser extendido por \texttt{CU Logueandose en el sistema}}{}
  \ucstep{Fin caso de uso.}{}
\end{usecasesteps}

\begin{usecase}
  \ucname{Logueandose en el Sistema}
  \ucpre{True}
  \ucpost{El pasajero queda logueado al sistema}
  \ucactor{Pasajero}
\end{usecase}
\begin{usecasesteps}
  \ucstep{El pasajero ingresa su nombre de pasajero y contraseña.}{}
  \ucstep{El sistema verifica que los datos sean validos.}{}
  \ucstep{Los datos son validos. El sistema muestra la pantalla principal.}
        {\ucalt{Los datos son invalidos, el sistema mostrar el mensaje de datos incorrectos. Ir a 1.}}
  \ucstep{Fin caso de uso.}{}
\end{usecasesteps}

\begin{usecase}
  \ucname{Solicitando un viaje}
  \ucpre{El pasajero esta logueado al sistema}
  \ucpost{Queda un nuevo viaje en estado pendiente(Ver modelo conceptual)}
  \ucactor{Pasajero}
\end{usecase}
\begin{usecasesteps}
  \ucstep{El pasajero solicita un viaje ingresando una direccion origen, una direccion destino, y un horario.}{}
  \ucstep{Los datos ingresados son correctos. El sistema computa una lista de taxistas apropiados y disponibles y la muestra en pantalla.}
        {\ucalt{El sistema no encuentra ningun taxista disponible y apropiado, el sistema muestra el error en pantalla. Ir a fin CU.
        Los datos ingresados no son validos, el sistema muestra un mensaje informando el error. Ir a 1.}}
  \ucstep{El pasajero selecciona un taxista de la lista(\texttt{CU Eligiendo Taxista y Taxi}).}{}
  \ucstep{El taxista elegido no rechaza el viaje. El sistema crea un nuevo viaje pendiente(Ver modelo conceptual) y notifica al Pasajero con un mensaje.}{\ucalt{El taxista elegido rechaza el viaje, el sistema asigna al siguiente taxista en orden de prioridad(distancia y calificacion). En caso de ser asignado otro taxista, ir al siguiente paso. Caso contrario, no se puede encontrar un taxista para el viaje, el sistema muestra el error en pantalla. Ir a Fin CU.}}
  \ucstep{Puede ser extendido por el \texttt{CU Cancelando Viaje} y \texttt{CU Consultando posicion del taxi}}{}
  \ucstep{Incluye al \texttt{CU Eligiendo Taxista y Taxi}}{}
  \ucstep{Fin caso de uso.}{}
\end{usecasesteps}

\begin{usecase}
  \ucname{Reservando un viaje}
  \ucpre{El pasajero esta logueado al sistema}
  \ucpost{Queda una nueva reserva asentada en el sistema, con un viaje pendiente asociado en estado pendiente(Ver modelo conceptual)}
  \ucactor{Pasajero}
\end{usecase}
\begin{usecasesteps}
  \ucstep{El pasajero solicita una reserva por un viaje ingresando los datos requeridos(Ver descripcion del sistema) y selecciona un taxista de la lista(\texttt{CU Eligiendo Taxista y Taxi}) que sera tenido en cuenta como preferencia principal para el viaje.}{}
  \ucstep{Los datos ingresados son validos. El sistema asenta una reserva con un viaje asociado, teniendo en cuenta la preferencia del taxista en el viaje y notifica al Pasajero con un mensaje.}{\ucalt{Los datos ingresados no son validos. El sistema informa esto en pantalla mostrando un mensaje. Ir a 1.}}
  \ucstep{El sistema pregunta al pasajero si desea pagar con el sistema automatizado de pago electronico(SAPE) o pagara el viaje en efectivo al finalizar el mismo.}{}
  \ucstep{El pasajero selecciona pago en efectivo al finalizar el viaje. Ir a Fin CU. Si el pasajero selecciona pagar con SAPE, ir al paso siguiente.}{}
  \ucstep{Usar \texttt{CU Debitando Saldo}}{}
  \ucstep{Puede ser extendido por el \texttt{CU Debitando saldo}}{}
  \ucstep{Incluye al \texttt{CU Eligiendo Taxista y Taxi}}{}
  \ucstep{Fin caso de uso.}{}
\end{usecasesteps}
\texttt{Contribuye al requerimiento \ref{req:R6}}{}\\
\textbf{Nota adicional: }\texttt{ Notar que aunque el viaje asociado a la reserva tenga a este taxista, puede cambiar posteriormente al momento de efectuarse el viaje debido a disponibilidad del taxista.}

\begin{usecase}
  \ucname{Eligiendo Taxista y taxi}
  \ucpre{El pasajero esta logueado al sistema}
  \ucpost{El pasajero elige taxista y taxi}
  \ucactor{Pasajero}
\end{usecase}
\begin{usecasesteps}
  \ucstep{El pasajero selecciona un taxista y taxi de la lista provista por el sistema}{}
  \ucstep{Fin caso de uso.}{}
\end{usecasesteps}
\texttt{Contribuye al requerimiento \ref{req:R3}}{}

\begin{usecase}
  \ucname{Cancelando Viaje}
  \ucpre{El pasajero esta logueado y existe un viaje pendiente a su nombre}
  \ucpost{El viaje es cancelado}
  \ucactor{Pasajero, Taxista}
\end{usecase}
\begin{usecasesteps}
  \ucstep{El pasajero ingresa al menu de viajes pendientes y solicita cancelacion del viaje en cuestion.}{}
  \ucstep{El sistema marca el viaje como cancelado e informa al taxista asociado de ser necesario.}{}
  \ucstep{Fin caso de uso.}{}
\end{usecasesteps}

\begin{usecase}
  \ucname{Cancelando Reserva}
  \ucpre{El pasajero esta logueado y existe una reserva pendiente a su nombre}
  \ucpost{La reserva y el viaje asociado a la misma, son cancelados}
  \ucactor{Pasajero}
\end{usecase}
\begin{usecasesteps}
  \ucstep{El pasajero ingresa al menu de reservas pendientes y solicita cancelacion de la reserva en cuestion.}{}
  \ucstep{El sistema marca el viaje asociado a la reserva como cancelado.}{}
  \ucstep{Fin caso de uso.}{}
\end{usecasesteps}

\begin{usecase}
  \ucname{Calificando al taxista}
  \ucpre{El Pasajero realizo un viaje y aun no califico al taxista de dicho viaje $\land$ el pasajero esta logueado al sistema}
  \ucpost{La calificacion es asociada al viaje de dicho taxista y esto impacta en la calificacion del taxista}
  \ucactor{Pasajero}
\end{usecase}
\begin{usecasesteps}
  \ucstep{El pasajero ingresa al menu de viajes, selecciona el viaje sin puntuar y asigna una calificacion en el rango $[0..10]$ al taxista con el que realizo el viaje.}{}
  \ucstep{El sistema valida la calificacion ingresada.}{}
  \ucstep{La calificacion es valida, el sistema asigna la calificacion y marca el viaje como puntuado, quitandolo de la lista de viajes sin puntuar.}{\ucalt{La calificacion no es valida, el sistema muestra mensaje de error en pantalla. Ir a 1.}}
  \ucstep{Fin caso de uso.}{}
\end{usecasesteps}
\texttt{Contribuye al requerimiento \ref{req:R2}}{}

\begin{usecase}
  \ucname{Consultando posicion del taxi}
  \ucpre{El pasajero esta logueado $\land$ el pasajero tiene un viaje pendiente}
  \ucpost{El pasajero conoce posicion del taxi}
  \ucactor{Pasajero}
\end{usecase}
\begin{usecasesteps}
  \ucstep{El pasajero ingresa al menu de viajes y solicita conocer la posicion del taxi asignado a su viaje pendiente}{}
  \ucstep{El sistema informa en un mapa en pantalla la posicion del taxi asignado al viaje y da un estimado de tiempo restante.}{}
  \ucstep{Fin caso de uso.}{}
\end{usecasesteps}
\texttt{Contribuye al requerimiento \ref{req:R1}}{}

\begin{usecase}
  \ucname{Debitando Saldo}
  \ucpre{El pasajero esta logueado y en un contexto de pago de reserva}
  \ucpost{El sistema debita el saldo de la cuenta del pasajero}
  \ucactor{Sistema automatizado de pago electronico(SAPE), Pasajero}
\end{usecase}
\begin{usecasesteps}
  \ucstep{El pasajero ingresa los datos de su cuenta para que el sistema (usando SAPE) pueda debitar el pago.}{}
  \ucstep{El sistema valida los datos con el sistema SAPE y verifica que haya saldo suficiente en la cuenta. }{}
  \ucstep{El sistema SAPE confirma que los datos son validos y hay saldo en la cuenta. El sistema procede a debitar el saldo enviando la orden al sistema SAPE.}{\ucalt{El sistema SAPE informa que los datos no son validos, el sistema informa en pantalla del error. Ir a 1.}}
  \ucstep{El sistema SAPE informa al sistema que la transaccion fue correcta. El sistema muestra en pantalla que la operacion fue satisfactoria.}{\ucalt{El sistema SAPE informa que hubo un error, o hubo un error de conectividad entre el sistema y el sistema SAPE. El sistema muestra el error en pantalla. Ir a 1.}}
  \ucstep{Fin caso de uso.}{}
\end{usecasesteps}

\begin{usecase}
  \ucname{Actualizando Posicion de Taxista}
  \ucpre{La operadora esta logueada}
  \ucpost{El sistema tiene una posicion actualizada del Taxista}
  \ucactor{Operadora}
\end{usecase}
\begin{usecasesteps}
  \ucstep{La operadora vuelca en el sistema una posicion informada por radio por algun taxista}{}
  \ucstep{Fin caso de uso.}{}
\end{usecasesteps}
\texttt{Contribuye al requerimiento \ref{req:R1}}{}

\begin{usecase}
  \ucname{Reportando cambio de posicion de taxi}
  \ucpre{-}
  \ucpost{El sistema tiene una posicion actualizada del taxi}
  \ucactor{Sistema de localizacion GPS}
\end{usecase}
\begin{usecasesteps}
  \ucstep{El sistema de GPS informa al sistema una posicion actualizada del taxi}{}
  \ucstep{Fin caso de uso.}{}
\end{usecasesteps}

\textbf{Nota:} Los \texttt{CU Personificando al Pasajero} y \texttt{Personificando al Taxista} se refieren a la interaccion de la operadora cuando no hay internet(o el pasajero no utiliza la interfaz web) para realizar todos los casos de uso pertenecientes a \texttt{Pasajero}(via telefono) y \texttt{Taxista}(via radio).

\begin{usecase}
  \ucname{Personificando al Pasajero}
  \ucpre{La operadora esta logueada}
  \ucpost{El Pasajero, logra, por medio de la operadora realizar la accion requerida}
  \ucactor{Operadora, Pasajero}
\end{usecase}
\begin{usecasesteps}
  \ucstep{El pasajero quiere realizar una operacion y se comunica con la operadora por telefono}{}
  \ucstep{La operadora le solicita los datos de acceso a su cuenta.}{}
  \ucstep{El pasajero tiene cuenta, se la informa a la operadora. La operadora valida la identidad del pasajero.}{\ucalt{El pasajero no tiene cuenta. La operadora le solicita los datos y lo registra en el sistema. Ir al paso siguiente.}}
  \ucstep{El pasajero le informa a la operadora que operacion quiere realizar y la operadora lo realiza en su nombre.}{}
  \ucstep{Fin caso de uso.}{}
\end{usecasesteps}
\texttt{Contribuye al requerimiento \ref{req:R5}}{}

\begin{usecase}
  \ucname{Personificando al Taxista}
  \ucpre{La operadora esta logueada}
  \ucpost{El Taxista, logra, por medio de la operadora realizar la accion requerida}
  \ucactor{Operadora, Taxista}
\end{usecase}
\begin{usecasesteps}
  \ucstep{El taxista quiere realizar una operacion y se comunica con la operadora via radio}{}
  \ucstep{El taxista le informa a la operadora que operacion quiere realizar y la operadora lo realiza en su nombre.}{}
  \ucstep{Puede ser extendido por el \texttt{CU Actualizando Posicion de Taxista}.}{}
  \ucstep{Fin caso de uso.}{}
\end{usecasesteps}

\begin{usecase}
  \ucname{Rechazando un viaje}
  \ucpre{El taxista esta logueado}
  \ucpost{El viaje ofrecido al taxista no tiene asignado a este taxista}
  \ucactor{Taxista}
\end{usecase}
\begin{usecasesteps}
  \ucstep{El sistema le ofrece al taxista un viaje.}{}
  \ucstep{El taxista lo rechaza desde su interfaz de tableta movil.}{}
  \ucstep{Fin caso de uso.}{}
\end{usecasesteps}

\begin{usecase}
  \ucname{Reportando un viaje finalizado}
  \ucpre{El taxista esta logueado $\land$ el viaje en cuestion esta en estado en progreso(Ver FSM Ciclo de Vida Viaje)}
  \ucpost{El viaje queda en estado finalizado}
  \ucactor{Taxista}
\end{usecase}
\begin{usecasesteps}
  \ucstep{El taxista marca un viaje como \texttt{Finalizado.}}{}
  \ucstep{El sistema asigna el estado de este viaje a finalizado.}{}
  \ucstep{Si el pasajero paga en efectivo, extiende el \texttt{CU Reportando el pago de un viaje}}{}
  \ucstep{Fin caso de uso.}{}
\end{usecasesteps}

\begin{usecase}
  \ucname{Reportando un viaje como \texttt{En Progreso}}
  \ucpre{El taxista esta logueado $\land$ el viaje en cuestion esta en estado pendiente(Ver FSM Ciclo de Vida Viaje)}
  \ucpost{El viaje queda en estado en progreso}
  \ucactor{Taxista}
\end{usecase}
\begin{usecasesteps}
  \ucstep{El taxista marca un viaje como \texttt{En progreso}.}{}
  \ucstep{El sistema asigna el estado de este viaje a \texttt{En progreso}.}{}
  \ucstep{Fin caso de uso.}{}
\end{usecasesteps}

\begin{usecase}
  \ucname{Recibiendo un viaje}
  \ucpre{El taxista esta logueado $\land$ el viaje en cuestion esta en estado pendiente(Ver FSM Ciclo de Vida Viaje)}
  \ucpost{Dependiendo si el taxista acepta o no el viaje, es asignado o no a el mismo}
  \ucactor{Taxista}
\end{usecase}
\begin{usecasesteps}
  \ucstep{El sistema le ofrece al taxista un nuevo viaje en su interfaz de tableta.}{}
  \ucstep{El taxista acepta el viaje. Ir al paso siguiente}{\ucalt{El taxista no acepta el viaje. El sistema intenta asignar el viaje a otro taxista. Ir a Fin CU}}
  \ucstep{El sistema asigna el taxista de este viaje a el taxista en cuestion.}{}
  \ucstep{Fin caso de uso.}{}
\end{usecasesteps}
\texttt{Contribuye al requerimiento \ref{req:R4}}{}

\begin{usecase}
  \ucname{Reportando el pago de un viaje}
  \ucpre{El taxista esta logueado $\land$ el viaje en cuestion esta en estado en progreso(Ver FSM Ciclo de Vida Viaje)}
  \ucpost{El sistema conoce el monto pagado por ese viaje}
  \ucactor{Taxista}
\end{usecase}
\begin{usecasesteps}
  \ucstep{El taxista ingresa el monto recibido al sistema.}{}
  \ucstep{El sistema guarda el valor el taxista de este viaje a el taxista en cuestion.}{}
  \ucstep{Fin caso de uso.}{}
\end{usecasesteps}

\begin{usecase}
  \ucname{Consultando calificaciones de taxistas}
  \ucpre{-}
  \ucpost{El directivo conoce las calificaciones}
  \ucactor{Directivo}
\end{usecase}
\begin{usecasesteps}
  \ucstep{El directivo accede en la intranet del sistema, al menu de estadisticas y selecciona \texttt{Calificaciones de taxistas}.}{}
  \ucstep{El sistema muestra en pantalla la informacion de calificaciones de taxistas.}{}
  \ucstep{Fin caso de uso.}{}
\end{usecasesteps}
\texttt{Contribuye al requerimiento \ref{req:R10}}{}

\begin{usecase}
  \ucname{Consultando facturacion}
  \ucpre{-}
  \ucpost{El directivo conoce la facturacion}
  \ucactor{Directivo}
\end{usecase}
\begin{usecasesteps}
  \ucstep{El directivo accede en la intranet de estadisticas, al menu de estadisticas y selecciona \texttt{Facturacion}.}{}
  \ucstep{El sistema muestra en pantalla la informacion de facturacion.}{}
  \ucstep{Fin caso de uso.}{}
\end{usecasesteps}
\texttt{Contribuye al requerimiento \ref{req:R8}}{}

\begin{usecase}
  \ucname{Consultando viajes realizados}
  \ucpre{-}
  \ucpost{El directivo conoce la lista de viajes realizados}
  \ucactor{Directivo}
\end{usecase}
\begin{usecasesteps}
  \ucstep{El directivo accede en la intranet del sistema, al menu de estadisticas y selecciona \texttt{Viajes realizados}.}{}
  \ucstep{El sistema muestra en pantalla la informacion de Viajes realizados.}{}
  \ucstep{Fin caso de uso.}{}
\end{usecasesteps}
\texttt{Contribuye al requerimiento \ref{req:R9}}{}

\begin{usecase}
  \ucname{Cambiando su estado}
  \ucpre{El taxista esta logueado}
  \ucpost{El estado del taxista queda modificado de acuerdo a lo solicitado.}
  \ucactor{Taxista}
\end{usecase}
\begin{usecasesteps}
  \ucstep{El taxista solicita un cambio de estado al sistema(Ver Modelo conceptual, EstadoTaxista).}{}
  \ucstep{El sistema cambia el estado del taxista de acuerdo al nuevo estado solicitado.}{}
  \ucstep{Fin caso de uso.}{}
\end{usecasesteps}

\begin{usecase}
  \ucname{Cancelando viaje asignado (Taxista)}
  \ucpre{El taxista esta logueado $\land$ tiene un viaje asignado con estado pendiente}
  \ucpost{El viaje asignado al taxista se desliga de este taxista y el pasajero es notificado}
  \ucactor{Taxista, Pasajero}
\end{usecase}
\begin{usecasesteps}
  \ucstep{El taxista selecciona un viaje pendiente y solicita cancelarlo.}{}
  \ucstep{El sistema refleja este cambio y notifica al Pasajero.}{}
  \ucstep{Fin caso de uso.}{}
\end{usecasesteps}